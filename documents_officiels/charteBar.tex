%%%%%%%%%%%%%%%%%%%%%%%%%%%%%%%%%%%%%%%%%%%%%%%%%%%%%%%%%%%%%%%%%%%%%%%
%                                                                     %
%                        Charte du club Bar                           %
%                                                                Beta %
%%%%%%%%%%%%%%%%%%%%%%%%%%%%%%%%%%%%%%%%%%%%%%%%%%%%%%%%%%%%%%%%%%%%%%%
%
%  Auteur :
%    Julien Déoux <julien.deoux@telecomnancy.net>
%    
%  Licence :
%    CC-BY-NC 4.0 (http://creativecommons.org/licenses/by-nc/4.0)
%
%%%



%%%%%%%%%%%%%%%%%%%
%  Configuration  %
%%%%%%%%%%%%%%%%%%%

\documentclass{article} % Jusqu'à preuve du contraire, les documents édités par le
                                                     % Ceten ne font pas 300 pages
\usepackage[a4paper,includeheadfoot,margin=2.54cm]{geometry}
\usepackage[francais]{babel}

\usepackage{hyperref}
\usepackage{graphicx} 
\usepackage{titlesec} 
\usepackage{ifxetex} 
\usepackage[usenames]{xcolor} 
\definecolor{newCeten}{RGB}{130,11,95}
\usepackage{enumitem,amssymb}
\newlist{todolist}{itemize}{2}
\setlist[todolist]{label=$\square$}
\usepackage{multicol}
\usepackage{setspace}
\usepackage{fontspec} 
\usepackage[all]{nowidow}

\setmainfont{Roboto Slab}
\setsansfont{Roboto}
\setmonofont{Roboto Mono}
\newfontfamily\condensed{Roboto Condensed}
\newfontfamily\condensedlight{Roboto Condensed Light}
\newfontfamily\light{Roboto Slab Light}
\titleformat*{\section}{\large\condensedlight\color{newCeten}}

\title{Charte du club Bar}
\author{Julien Déoux}
\date\today

%%%%%%%%%%%%%%
%  Document  %
%%%%%%%%%%%%%%

\begin{document}
	
	%---------------%
	% Page de garde %
	%---------------%

	\begin{titlepage}
		\begin{center}
			\includegraphics[width=\textwidth]{images/ceten}\par
			\vspace{2cm}
			{\Huge \light{} Charte du club bar}\par
			\vfill
			\begin{spacing}{1.5}
				année: \underline{\hspace{2cm}}\\
				président: \underline{\hspace{8cm}}\\
				trésorier: \underline{\hspace{8cm}}\\
				secrétaire: \underline{\hspace{8cm}}\\
				vice-président: \underline{\hspace{8cm}}\\
				responsable logistique: \underline{\hspace{8cm}}\\
				responsable informatique: \underline{\hspace{8cm}}\\
				référent de TELECOM Nancy: \underline{\hspace{8cm}}\\
			\end{spacing}
			\vspace{\baselineskip}
			objet du club: Tenir le bar de l'école et vendre des produits
			alimentaires aux élèves
			\vfill
			{\footnotesize \light{} Ce texte régit le fonctionnement, les
			obligations et les droits attribués au club bar, club du Ceten comme
			défini dans le règlement intérieur. Un exemplaire est remis au
			président du club, l’autre est archivé au sein du BDE\@. \\
			Charte modifiée et votée le 12/03/2013 par le BDE du Ceten, avec
			l'accord du président du club bar}
		\end{center}
	\end{titlepage}

	%----------%
	% Articles %
	%----------%

	\pagenumbering{arabic}

	\begin{multicols}{2}

		\section{Constitution}
\label{sec:constitution}
		
		{\small

			Le club bar est une entité interne au Ceten, composé de membres et
			dédié à l'objet défini au début de cette charte. Le fonctionnement
			et les engagements pris par ce club sont définis dans la présente
			charte.

			L'existence de ce club est subordonné à la signature de la présente
			charte par le responsable des clubs du BDE, le référent de TELECOM
			Nancy et par les président, trésorier, secrétaire, vice-président,
			responsable logistique et responsable informatique du club.
			
			Faisant partie intégrante de l’association, toutes les règles et
			décisions s’imposant au Ceten s’appliquent également au club.

		}

		\section{Composition}
\label{sec:composition}

		{\small
		
			Le club bar est composé de membres qui doivent également être
			membres du Ceten. Ce club se compose au minimum de six personnes
			distinctes qui sont:
			\begin{itemize}
				\item le président, dont le rôle est défini en
					Section~\ref{sec:presidence}
				\item le trésorier, dont le rôle est défini en
					Section~\ref{sec:tresorerie}
				\item le secrétaire, dont le rôle est défini en
					Section~\ref{sec:secretariat}
				\item le vice-président, qui soutient le président dans ses
					fonctions
				\item le responsable logistique, qui surveille l'état des stocks
					et le bon déroulement des approvisionnements du club
				\item le responsable informatique, qui s'assure du bon
					fonctionnement des matériels et logiciels informatiques
					nécessaires au fonctionnement du club
			\end{itemize}

			Ces six personnes déterminent le «bureau restreint» du club, qui a
			pour fonction de gérer l'ensemble du club. Le club n'a lieu
			d'exister sans la présence de ces six personnes.

			La structure interne du club et laissée libre au bureau restreint.
			En particulier, il peut désigner un «bureau complémentaire» dont le
			rôle est d'aider le bureau restreint à assurer ses fonctions, et
			dont la constitution doit être immédiatement transmise au
			responsable des clubs du BDE\@.

			L'ensemble de ces deux bureaux constitue le «bureau étendu» du club.
			Dans la suite de cette charte, l'appellation «bureau» désigne le
			bureau étendu du club.

		}

		\section{Durée}
\label{sec:duree}

		{\small
		
			Un mandat du club bar commence au 1\up{er} janvier et se termine au
			31 décembre de l'année spécifiée au début de la charte, sauf en cas
			de réélection d'un ou plusieurs membres du bureau restreint du club.

			Un membre du club l'est de la date de son inscription jusqu'au
			1\up{er} septembre suivant.

		}

		\section{Activité au sein du club}
\label{sec:activite}
		
		{\small
		
			Le club se limite aux activités décrites par l’objet défini au début
			de la charte et uniquement par celui-ci. Il mettra en œuvre les
			moyens humains et logistiques pour parvenir au mieux à la
			réalisation de ces activités. Toute activité n’ayant pas de relation
			directe avec l’objet du club doit faire avant réalisation une
			demande auprès du BDE\@. Le BDE aidera, dans la mesure du possible
			et du raisonnable, le club à réaliser ses activités, par un
			financement, du matériel, des formations et un suivi.

		}

		\section{Prise de décisions}
\label{sec:decisions}

		{\small
		
			Lors de son exercice, le bureau du bar est régulièrement amené à
			prendre des décisions sur le fonctionnement du club. Lorsque
			celles-ci concernent la politique des prix du club, la gestion de
			ses membres ou la composition de son bureau, elles doivent d'abord
			faire l'objet d'une réunion avec compte-rendu écrit.

			Le BDE est invité permanent des réunions du bureau. La présence de
			membres du club ou de personnes extérieures au club est laissée à la
			discrétion du président. Le BDE se réserve le droit de refuser la
			présence en réunion d'une personne extérieure au Ceten. Chaque
			compte-rendu doit être envoyé à l'ensemble du bureau ainsi qu'au
			BDE\@ dans les trois jours suivant la réunion.
		
		}

		\section{Propriété du club}
\label{sec:propriete}
		
		{\small
		
			Aucun bien ne peut appartenir au club. Cependant, des biens sont
			confiés par le BDE au club en fonction de ses besoins. Le président
			du club est responsable des biens prêtés par le BDE\@. La liste des
			biens prêtés devra être établie entre le responsable logistique du
			club et celui du BDE, et être mise à jour régulièrement.

			Le BDE doit, dans la mesure des fonds et matériels disponibles,
			fournir au club l’ensemble des moyens nécessaires à la réalisation
			de ses activités. Ces biens peuvent être utilisés par d’autres
			membres du Ceten que ceux composant le club, cependant cette
			utilisation doit être soumise à l’approbation du président du club.

			Le BDE se réserve, à titre exceptionnel, le droit d’utiliser un des
			biens confié au club en prévenant son président. Le club bar se voit
			également attribuer un local à accès exclusif. Cet accès est réservé
			au bureau du club et au BDE\@. L’accès à ces locaux aux autres
			membres du club est laissé à la discrétion du bureau du club et du
			BDE\@.
			
		}

		\section{Présidence}
\label{sec:presidence}

		\section{Trésorerie}
\label{sec:tresorerie}

		\section{Secrétariat}
\label{sec:secretariat}

	\end{multicols}

\end{document}
