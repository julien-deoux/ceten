%%%%%%%%%%%%%%%%%%%%%%%%%%%%%%%%%%%%%%%%%%%%%%%%%%%%%%%%%%%%%%%%%%%%%%%
%                                                                     %
%                        Charte des clubs                             %
%                                                                Beta %
%%%%%%%%%%%%%%%%%%%%%%%%%%%%%%%%%%%%%%%%%%%%%%%%%%%%%%%%%%%%%%%%%%%%%%%
%
%  Auteur :
%    Julien Déoux <julien.deoux@telecomnancy.net>
%    
%  Licence :
%    CC-BY-NC 4.0 (http://creativecommons.org/licenses/by-nc/4.0)
%
%%%



%%%%%%%%%%%%%%%%%%%
%  Configuration  %
%%%%%%%%%%%%%%%%%%%

\documentclass{article} % Jusqu'à preuve du contraire, les documents édités par le
                                                     % Ceten ne font pas 300 pages
\usepackage[a4paper,includeheadfoot,margin=2.54cm]{geometry}
\usepackage[francais]{babel}

\usepackage{hyperref}
\usepackage{graphicx} 
\usepackage{titlesec} 
\usepackage{ifxetex} 
\usepackage[usenames]{xcolor} 
\definecolor{newCeten}{RGB}{130,11,95}
\usepackage{enumitem,amssymb}
\newlist{todolist}{itemize}{2}
\setlist[todolist]{label=$\square$}
\usepackage{multicol}

\ifxetex
	\usepackage{fontspec} 
	\setmainfont{Roboto Slab}
	\setsansfont{Roboto}
	\setmonofont{Roboto Mono}
	\newfontfamily\condensed{Roboto Condensed}
	\newfontfamily\condensedlight{Roboto Condensed Light}
	\newfontfamily\light{Roboto Slab Light}
	\titleformat*{\section}{\large\condensedlight\color{newCeten}}
\else
	\usepackage[utf8]{inputenc} 
	\usepackage[T1]{fontenc} 
\fi

\title{Charte des clubs}
\author{Julien Déoux}
\date\today

%%%%%%%%%%%%%%
%  Document  %
%%%%%%%%%%%%%%

\begin{document}

	\pagenumbering{roman}

	%---------------%
	% Page de garde %
	%---------------%
	
	\begin{titlepage}
		\begin{center}
			\includegraphics[width=\textwidth]{images/ceten.png}\par
			\vspace{1.5cm}
			{\Huge \light Charte des clubs}\par
			\vspace{1.5cm}
		\end{center}
		\begin{center}
			Club : \underline{\hspace{8cm}}
			Année : \underline{\hspace{1.5cm}}\\
			~\\
			$\square$ Création du club
			\hspace{3cm}
			$\square$ Renouvellement du bureau du club\\
			~\\
			Président du club : \underline{\hspace{7cm}}\\
			Trésorier du club : \underline{\hspace{7cm}}\\
			Secrétaire du club : \underline{\hspace{7cm}}\\
			Référent de TELECOM Nancy : \underline{\hspace{7cm}}\\
		\end{center}
		Type du club :
		\begin{todolist}
			\item Club de services
			\item Club de loisirs
			\item Club évènementiel (Mois de passation :
				\underline{\hspace{5cm}})
		\end{todolist}
		~\\
		Objet du club :\\
		\underline{\hspace{\textwidth}}
		\underline{\hspace{\textwidth}}
		\underline{\hspace{\textwidth}}
		\underline{\hspace{\textwidth}}
		\underline{\hspace{\textwidth}}
 
		\vfill
		\begin{center}
			{\light Ce texte régit le fonctionnement, les obligations et les
			droits attribués aux différents Clubs composant le CETEN, comme
			définis dans les statuts. Un exemplaire est remis au président du
			club, l’autre est archivé au sein du BDE. \\
			Charte modifiée et votée le 12/03/2013 par le BDE du CETEN}
		\end{center}
	\end{titlepage}

	%--------------------%
	% Table des matières %
	%--------------------%
	
	\tableofcontents
	\clearpage

	%----------%
	% Articles %
	%----------%

	\pagenumbering{arabic}

	\begin{multicols}{2}
		
		\section{Définition}
			
		{\small
			
			Un "club" est une entité interne au CETEN, composée de membres et
			dédiée à un but ou une activité particulière. Le fonctionnement et
			les engagements pris par le club sont détaillés dans cette Charte
			des Clubs. L’objet spécifique du club est défini au début de cette
			charte.

			Un club n’a existence que par signature de cette présente charte par
			le Responsable des clubs du BDE, la personne référent de TELECOM
			Nancy et par les président, trésorier et secrétaire du club.

			Faisant partie intégrante de l’association, toutes les règles et
			décisions s’imposant au CETEN s’appliquent également au Club.
			
			}

		\section{Type}
			
		{\small
		
			Il est défini trois types de Clubs au sein du CETEN :
			\begin{itemize}
				\item Les clubs de type évènementiel, dont le but est
					d’organiser ou de participer à une ou des manifestations
					ponctuelles
				\item Les clubs de type services, qui apportent un service
					particulier dans un domaine précis aux membres du CETEN
				\item Les clubs de type loisirs, regroupant les autres clubs et
					qui sont généralement axés sur le divertissement
			\end{itemize}
			Le type du club doit être défini avec le BDE lors de sa création et
			notifié au début de la charte.
			
		}
		
		\section{Durée}
		
		{\small
			
			Un club existe jusqu’à la fin de l’année civile spécifiée au début
			de la charte, ou jusqu’à dernier jour du mois de passation défini au
			début de la charte pour les clubs de type évènementiel. Cette durée
			est reconductible par renouvellement du club via cette charte.

			Un membre du club l’est jusqu’à la fin du mandat du bureau du club.
			% À voir
			
		}

		\section{Composition}
		
		{\small
		
			Un club est composé de membres qui doivent être également membres du
			CETEN. Un club se compose au minimum de trois personnes distinctes
			qui sont :
			\begin{itemize}
				\item le président du club, responsable de celui-ci
				\item le trésorier, chargé de tenir régulièrement une
					comptabilité précise du club
				\item le secrétaire, chargé de rédiger les comptes rendus des 
					réunions tenues par le club.
					% Suivi de la liste des membres
			\end{itemize}
			Ces trois personnes déterminent le "bureau restreint" du club, qui a
			pour fonction de gérer l’ensemble du club. Le club n’a lieu
			d’exister sans la présence de ces trois personnes. La structure
			interne du Club est laissée libre au Bureau.

			% Bureau étendu
			Tous les membres du Club devront au préalable avoir fait
			l’inscription à celui-ci via le formulaire fourni en Annexe 1, qui
			sera transmis au responsable Clubs du BDE.
			% À voir
			Chaque bureau de Club doit tenir une liste à jour des membres de son
			club et transmettre au BDE tout changement dans sa structure
			interne.
			
		}
		
		\section{Création}
		
		{\small
		
			La création d’un club se fait par trois membres du CETEN définissant
			leur rôle dans ledit club (président, trésorier et secrétaire). Le
			référent de TELECOM Nancy est désigné par la Direction des Études.
			Toute demande de création de club doit être soumise au vote du BDE.
			En cas d’accord et la charte des clubs signée, les fondateurs du
			club deviennent membres du club et définissent le bureau du club.
			
		}

		\section{Activité au sein du club}
		
		{\small
		
			Le club se limite aux activités décrites par l’objet défini au début
			de la charte et uniquement par celui-ci. Il mettra en œuvre les
			moyens humains et logistiques pour parvenir au mieux à la
			réalisation de ces activités. Toute activité n’ayant pas de relation
			directe avec l’objet du club doit faire avant réalisation une
			demande auprès du BDE. Le BDE aidera, dans la mesure du possible et
			du raisonnable, le club à réaliser ses activités, par un 
			financement, du matériel, des formations et un suivi.

			Pour s’assurer que le club exerce correctement son activité, un
			compte-rendu d’activités est demandé par le BDE tous les mois.
			
		}

		\section{Déroulement des réunions}
		
		{\small
		
			En période scolaire, chaque club doit se réunir à un minimum d’une
			fois par mois sur convocation du président ou à la demande du tiers
			de ses membres. Pour les clubs de type événementiel, deux mois avant
			l’évènement, cette période est réduite à une fois tous les quinze
			jours.  

			Seuls les membres du club peuvent assister aux réunions. Le bureau
			du club se réserve le droit d’inviter un membre du CETEN ou toute
			personne étrangère à l’association si celle-ci est susceptible de
			l’aider. Tous les membres du BDE sont invités permanents des
			réunions du club.

			La prise de décisions se fait à la majorité absolue des voix
			exprimées. En cas d’égalité, la voix du président est prépondérante.
			Toutes les délibérations feront l’objet d’une retranscription écrite
			mise à disposition des membres de l’association. Si une délibération
			est contestée par la majorité absolue des membres du club, celle-ci
			est annulée et doit être reconsidérée à la prochaine réunion du
			club.

			Chaque réunion fera l’objet d’un compte-rendu qui sera archivé par
			le bureau du club et transmis au BDE dans les trois jours suivant la
			réunion.
			
		}

		\section{Propriété du club}
		
		{\small
		
			Aucun bien ne peut appartenir au club. Cependant, des biens sont
			confiés par le BDE au club en fonction de ses besoins. Le président
			du Club est responsable des biens prêtés par le BDE. La liste des
			biens prêtés devra être établie entre le club et le Responsable
			logistique du BDE, et être mise à jour régulièrement.

			Le BDE doit, dans la mesure des fonds et matériels disponibles,
			fournir au club l’ensemble des moyens nécessaires à la réalisation
			de ses activités. Ces biens peuvent être utilisés par d’autres
			membres du CETEN que ceux composant le Club, cependant cette
			utilisation doit être soumise à l’approbation du président du Club.

			Le BDE se réserve, à titre exceptionnel, le droit d’utiliser un des
			biens confié au Club en prévenant son président. Un Club peut
			également se voir attribuer, sous décision du BDE, des locaux à
			accès exclusif ou partagés avec d’autres clubs. Cet accès est
			réservé aux bureaux des clubs concernés et au BDE. L’accès à ces
			locaux aux autres membres du club est laissé à la discrétion des
			bureaux des clubs concernés et au BDE.
			
		}

		\section{Comptabilité}

		{\small

			Le président et le trésorier sont tenus responsables de la
			comptabilité du Club. Ils doivent veiller au respect des
			budgets attribués par le BDE au début de l’année civile.
			
			Chaque Club se doit de préparer un budget prévisionnel de
			son activité. Il doit se faire avant l’établissement du budget
			prévisionnel du CETEN de l’année à venir. Ce budget peut-
			être nul si le Club ne prévoit aucune dépense et aucune
			recette durant l’année. Il y précisera les subventions
			demandées au BDE et les recettes reversées à l’association.
			Le budget prévisionnel de chaque Club doit être
			nécessairement équilibré.
			
			Une réunion avec l’ensemble des Clubs et le BDE est
			organisée chaque année pour répartir et valider l’ensemble
			des subventions attribuées à chacun des Clubs.
			
			À la fin de son mandat, le Club devra présenter un budget
			final (y détaillant l’ensemble des dépenses et recettes du
			Club), pour pouvoir l’inclure à celui du CETEN.

			L’état des dépenses et recettes du Club doit être géré par le
			trésorier du Club. Il tient informé régulièrement le trésorier
			du BDE des flux d’argent au sein du Club. L’utilisation d’un
			échéancier est encouragé pour savoir quelles recettes et
			dépenses ont été réellement faites. Chaque mouvement doit
			être justifié par une pièce comptable (factures, etc.). Toutes
			ces pièces comptables devront être archivées au sein du
			local du Club, ou au BDE si le Club ne possède pas de local.
			De plus, une copie numérique de ces pièces doit être
			transmise au BDE.

		}

		\section{Trésorerie}

		{\small

			Chaque Club peut, s’il le souhaite, posséder une caisse
			monnaie. Le trésorier doit tenir à jour un journal de caisse
			enregistrant les entrées/sorties d’argent de la caisse du
			Club. Ce journal de caisse devra être remis en fin d’année au
			trésorier du BDE.

			Toute transaction nécessitant l’utilisation du compte
			bancaire du BDE (paiement sur internet, émission ou
			encaissement d’un chèque, etc.) doit obligatoirement être
			faite par le trésorier du BDE. Néanmoins, le trésorier du Club
			indique dans sa comptabilité la dépense ou la recette
			effectuée (même si celle-ci provient ou est perçue par le BDE
			et non par le Club).

			Un membre du Club peut fournir une avance pour un achat
			du Club. Le remboursement est validé et effectué par le BDE
			sous présentation d’une facture justifiant la dépense
			occasionnée et par la signature de la feuille de
			remboursement fourni en annexe.

		}

		\section{Représentation du CETEN}

		{\small

			Le Club n’est en aucun cas une entité morale. Elle
			représente uniquement le CETEN pour l’activité décrite en
			objet de la charte et ne peut s’accorder d’aucune
			responsabilité de l’association.

			Toute demande de sponsors, devis, accord, et autre contrat
			écrit entre le Club et une personne morale doit être
			explicitement faite au nom du CETEN. Seules les personnes
			compétentes du BDE possèdent les droits de signature. La
			responsabilité est alors portée sur l’association.

		}

		\section{Election du nouveau bureau}

		{\small

			Tout Club est responsable du renouvellement du bureau et
			de son élection. Peuvent se présenter au Bureau du Club
			uniquement les membres qui ont adhéré au Club au moins
			30 jours avant le scrutin pour les clubs de Service/Loisirs, et
			les personnes ayant participé à l’organisation d’au moins un
			évènement du club pour les clubs de type Evènementiel.
			Chaque élection du bureau doit se faire en présence d’au
			moins un représentant du BDE.

			Le vote se fait à main levée. Un vote à bulletin secret peut
			être admis si la majorité des membres présents ou le Bureau
			du Club le souhaite. Le vote par procuration est admis dans
			la limite d’une procuration par personne.

			La passation doit se faire durant le dernier mois du mandat
			du Club ou dans le mois de passation spécifié au début de la
			charte pour les Clubs de type Évènementiel.

			Pour les Clubs de type Services et Évènementiel, le scrutin
			est ouvert à tous les membres du CETEN. Pour les Clubs de
			type Loisirs, le scrutin est réservé aux uniques membres du
			Club, qui ont dû adhérer à celui-ci au moins 30 jours avant
			l’élection.

			L’élection se fait par poste (président, trésorier, secrétaire)
			en deux tours. Pour chaque poste, est élu au premier tour la
			personne ayant obtenu la majorité absolue des suffrages. Si
			aucune personne ne remporte la majorité absolue des voix
			lors du premier tour, un second tour est organisé à la
			majorité relative avec les deux personnes ayant reçu le plus
			de voix au premier tour. En cas d’égalité des voix, c’est au
			président sortant du Club de départager.

			Le nouveau bureau prend ses fonctions, après signature de
			la charte, le 1er Janvier de l’année civile suivante. Pour les
			Clubs de type Évènementiel, la date de prise de fonction est
			définie à la date de signature de la charte par le nouveau
			bureau.

			L’élection du nouveau bureau doit faire, comme pour toute
			réunion du Club, l’objet d’un Compte-Rendu.

		}
		
		\section{Passation du Club}

		{\small

			Durant le mois suivant l’élection, hors vacances scolaires,
			l’ancien bureau du Club doit assister le nouveau bureau pour
			sa prise de fonctions future.

			Un dossier de passation écrit doit être présent dans tous les
			Clubs. Celui-ci contiendra toutes les informations
			nécessaires pour la bonne continuité du Club. Il devra être
			enrichi au fur et à mesure par les mandats des différents
			bureaux.

			De même, le bureau sortant doit désigner en accord avec le
			nouveau un Responsable Passation qui sera l’interlocuteur
			privilégié du nouveau bureau avec l’ancien. Son rôle est
			d’assister le nouveau bureau dans sa prise de fonctions.

		}
		
		\section{Référent ALISE}

		{\small

			Pour certains Clubs à grande importance au sein du CETEN,
			l’association ALISE peut mettre à disposition du Club un ou
			plusieurs référents. Ayant déjà eu une expérience au sein de
			ce Club, ils peuvent apporter des conseils et éviter de
			reproduire les erreurs commises dans le passé. Le Club est
			tenu d’échanger régulièrement avec le référent ALISE sur
			son activité. Tous les Comptes Rendus des réunions devront
			également être envoyés au référent ALISE.

		}
		
		\section{Points CIPA et référent de TELECOM Nancy}

		{\small

			La participation à un Club peut donner un certain nombre de
			points CIPA pour l’obtention du diplôme de TELECOM
			Nancy. Il en revient à la direction de TELECOM Nancy de
			définir, dans le fascicule 0A, le nombre de points maximum
			attribués à la participation du Club en fonction de
			l’importance du Club, de son type, et de la fonction prise par
			les membres.

			Pour se voir attribuer les points CIPA, le Club possède un
			référent pouvant être une personne enseignante ou un
			membre de l’équipe d’animation de TELECOM Nancy. Le
			référent a pour rôle de conseiller le bureau dans la gestion
			de son Club, dans ses activités et est l’interlocuteur privilégié
			pour les questions de fonctionnement liées à
			l’administration. Celui-ci doit signer cette présente charte
			pour que l’attribution des points CIPA puisse être validée.
			Un membre enseignant ou administratif de TELECOM Nancy
			peut être référent d’un ou plusieurs Clubs. Pour permettre
			au référent de suivre son activité, le Club devra lui envoyer
			un compte-rendu d’activité au minimum deux fois par an.

			Dans le dernier mois du mandat du Club, le président du
			Club doit soumettre au référent, au directeur des études et
			ainsi qu’au BDE un compte-rendu de l’année du Club
			comprenant :

			\begin{itemize}
				\item Un bilan moral
				\item Un bilan financier
				\item La liste des membres actifs du Club en y précisant
					leur fonction et le travail réalisé par ceux-ci lors de
					l’année. De plus, le président attribue à chaque
					membre un nombre de points CIPA compris entre
					zéro et le maximum autorisé par le fascicule 0A
					selon son implication. Les points du président du
					Club sont attribués par le BDE.
					Le BDE peut demander une révision du compte-rendu et du
					nombre de points CIPA s’il juge celui-ci non conforme à
					l’activité du Club.
			\end{itemize}

		}
		
		\section{Perte de la qualité de membre du club}

		{\small

			La qualité de membre du Club se perd par démission, perte
			de la qualité de membre du CETEN ou exclusion prononcée
			par le bureau du Club ou le BDE. Une demande d’exclusion
			d’un membre peut être demandée au BDE par les deux-tiers
			des membres du Club.

			Le bureau du Club ou le BDE peut prononcer l’exclusion d’un
			membre du Club pour motif grave. Le membre concerné
			sera convoqué par ledit bureau pour plaider sa cause. La
			décision devra être approuvée par les deux tiers du bureau.
			Le vote par procuration n’est pas permis.

			Si ce membre faisait partie du Bureau, le BDE devra mettre
			en place au sein du Club une élection pour désigner son
			remplaçant. Une fois l’élection faite, la Charte devra être
			signée par le nouveau bureau. Si ce poste ne peut être
			pourvu par une autre personne, le Club est dissout.

		}
		
	\end{multicols}

	\vfill
	Fait en deux exemplaires à Villers-lès-Nancy le \underline{\hspace{5cm}}
	\vfill

	\begin{multicols}{2}
		Le président du club \\
		Le secrétaire du club 
	\end{multicols}
		\vspace*{4cm}
	\begin{multicols}{2}
		Le trésorier du club \\
		Le référent de TELECOM Nancy 
	\end{multicols}
		\vspace*{4cm}
	\begin{multicols}{2}
		Le président du BDE \\
		Le responsable des clubs du BDE 
	\end{multicols}
		\vspace*{4cm}

\end{document}
